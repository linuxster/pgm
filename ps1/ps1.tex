\documentclass[11pt]{article}
\usepackage{fullpage}
\usepackage{fancyhdr}

\usepackage{amsmath}
\usepackage{amssymb}

% header
\fancyhead{}
\fancyfoot{}
\fancyfoot[C]{\thepage}
\fancyhead[R]{Daniel Foreman-Mackey}
\fancyhead[L]{Probabilistic Graphical Models --- Problem Set 1}
\pagestyle{fancy}
\setlength{\headsep}{20pt}

% shortcuts
\newcommand{\Eq}[1]{Equation (\ref{eq:#1})}
\newcommand{\eq}[1]{Equation (\ref{eq:#1})}
\newcommand{\eqlabel}[1]{\label{eq:#1}}

% commands
\newcommand{\pr}[1]{\ensuremath{p(#1)}}
\newcommand{\no}[1]{\ensuremath{\tilde{#1}}}

\begin{document}

% === Problem 1 ===

\section{Problem 1}

\paragraph{(a)} The joint distribution \pr{X,Y}

\begin{equation}
    \begin{array}{r|ccc}
        X=  & 0 & 1 & 2 \\\hline
        Y=0 & 1/16 & 0 & 0 \\
        1 & 1/8 & 1/8 & 0 \\
        2 & 1/16 & 1/4 & 1/16 \\
        3 & 0 & 1/8 & 1/8 \\
        4 & 0 & 0 & 1/16 \\
    \end{array}
\end{equation}

\paragraph{(b)} The marginals \pr{X} and \pr{Y}

\begin{equation}
    \begin{array}{r|ccc}
        X=  & 0 & 1 & 2 \\\hline
        \pr{X} & 1/4 & 1/2 & 1/4
    \end{array} \hspace{2cm}
    \begin{array}{r|ccccc}
        Y=  & 0 & 1 & 2 & 3 & 4 \\\hline
        \pr{Y} & 1/16 & 1/4 & 3/8 & 1/4 & 1/16
    \end{array}
\end{equation}

\paragraph{(c)} The conditionals \pr{X|Y} and \pr{Y|X}

\begin{equation}
    \begin{array}{r|ccc}
        X=  & 0 & 1 & 2 \\\hline
        Y=0 & 1 & 0 & 0 \\
        1 & 1/2 & 1/2 & 0 \\
        2 & 1/6 & 2/3 & 1/6 \\
        3 & 0 & 1/2 & 1/2 \\
        4 & 0 & 0 & 1 \\
    \end{array} \hspace{2cm}
    \begin{array}{r|ccc}
        X=  & 0 & 1 & 2 \\\hline
        Y=0 & 1/4 & 0 & 0 \\
        1 & 1/2 & 1/4 & 0 \\
        2 & 1/4 & 1/2 & 1/4 \\
        3 & 0 & 1/4 & 1/2 \\
        4 & 0 & 0 & 1/4 \\
    \end{array}
\end{equation}

\paragraph{(d)} The distribution of $Z = Y-X$, \pr{Z}

\begin{equation}
    \begin{array}{r|ccc}
        Z=  & 0 & 1 & 2 \\\hline
        \pr{Z} & 1/4 & 1/2 & 1/4
    \end{array}
\end{equation}

% === Problem 2 ===
\section{Problem 2}

The conditional probabilities implied by this situation are as follows:

\begin{itemize}

    \item{The probability of testing positive given that you have the disease
        is $\pr{t|d} = 0.99$.}
    \item{The probability of testing positive given that you \emph{don't}
        have the disease is $\pr{t|\no{d}} = 0.01$.}
    \item{The marginal probability of having the disease is only
        $\pr{d} = 10^{-4}$ and the probability of not having the disease
        is $\pr{\no{d}} = 1-10^{-4}$.}
    \item{Therefore, the marginal probability of testing positive is
        \begin{equation}
            \pr{t} = \pr{t|d} \, \pr{d} + \pr{t|\no{d}} \, \pr{\no{d}}
                   = 0.99 \times 10^{-4} + 0.01 \, (1-10^{-4})
                   = 100.98 \times 10^{-4}
        \end{equation}
    }

\end{itemize}

\noindent The value that the patient really cares about, though is the
probability that they have the disease given that they tested positive
\pr{d | t}. This --- by Bayes --- is
\begin{equation}
    \pr{d | t} = \frac{\pr{d} \, \pr{t | d}}{\pr{t}}
               = \frac{0.99 \times 10^{-4}}{100.98 \times 10^{-4}}
               \approx 0.0098 \ll 1.
\end{equation}

% === Problem 3 ===
\section{Problem 3}

% === Problem 4 ===
\section{Problem 4}

We are given the three statements

\begin{enumerate}
    \item{\label{s1}$\pr{A,B|C} = \pr{A|C}\,\pr{B|C}$}
    \item{\label{s2}$\pr{A|B,C} = \pr{A|C}$}
    \item{\label{s3}$\pr{B|A,C} = \pr{B|C}$}
\end{enumerate}

\noindent To see that statement \ref{s1} implies statement \ref{s2}, apply
the chain rule to find
\begin{equation}
    \pr{A|C}\,\pr{B|C} \stackrel{1}{=} \pr{A,B|C} = \pr{B|C} \, \pr{A | B,C}.
\end{equation}
Cancelling \pr{B|C} on both sides, we find statement 2. Therefore, it is
clear that statement 1 implies statement 2. Also, since we have only used
the chain rule, the inverse also applies. Specifically, applying the chain
rule to statement 2, we find
\begin{equation}
    \pr{A|B,C} = \frac{\pr{A,B|C}}{\pr{B|C}} \stackrel{2}{=} \pr{A|C}
        \to [\mathrm{Statement}\,1].
\end{equation}
Similarly, statement 1 implies statement 3 as follows
\begin{equation}
    \pr{A|C}\,\pr{B|C} \stackrel{1}{=} \pr{A,B|C} = \pr{A|C} \, \pr{B|A,C}
        \to [\mathrm{Statement}\,3]
\end{equation}
and the inverse
\begin{equation}
    \pr{B|A,C} = \frac{\pr{A,B|C}}{\pr{A|C}} \stackrel{3}{=} \pr{B|C}
        \to [\mathrm{Statement}\,1].
\end{equation}
Finally, since the equivalence holds between 1 and 2 and also between 1 and 3,
it is clear that 2 and 3 are also equivalent.


% === Problem 5 ===
\section{Problem 5}

\paragraph{(a)}

By Bayes' Theorem,
\begin{equation}
    \eqlabel{p51}
    \pr{H | E_1,E_2} = \frac{\pr{E_1, E_2|H} \, \pr{H}}{\pr{E_1,E_2}}.
\end{equation}
Therefore, set (ii) is clearly sufficient for this calculation. Without any
conditional independence assumptions, \eq{p51} cannot be simplified any
further so the other two sets are not sufficient. In particular,
$\pr{E_1, E_2|H} \ne \pr{E_1|H}\,\pr{E_2|H}$ unless $E_1 \perp E_2 | H$.

\paragraph{(b)}

Since $E_1 \perp E_2 | H$, $\pr{E_1, E_2|H} = \pr{E_1|H}\,\pr{E_2|H}$ and
\eq{p51} becomes
\begin{equation}
    \pr{H | E_1,E_2} = \frac{\pr{E_1|H}\, \pr{E_2|H} \, \pr{H}}{\pr{E_1,E_2}}.
\end{equation}
Therefore, sets (i) and (ii) are now sufficient. Set (iii) is not sufficient
because it would require that $E_1 \perp E_2$ but $E_1$ and $E_2$ are only
\emph{conditionally} independent.

% === Problem 6 ===
\section{Problem 6}

% === Problem 7 ===
\section{Problem 7}

% === Problem 8 ===
\section{Problem 8}

\paragraph{(a)}

The set $A$ is $\{ X_2, X_3, X_4, X_5, X_8 \}$. Clearly, all the nodes
that are directly connected to $X_1$ (i.e.~$\{ X_2, X_3, X_4, X_8 \}$) must
be included in $A$ because a direct connection always constitutes an active
path. The inclusion of $X_5$ is not immediately obvious but if we just look
at the part of the graph containing $X_5$, we find the V-structure
$X_1 \to X_3 \gets X_5$. If we condition on $X_3$ (which we will do because
it is one of the directly connected nodes), it couples its parents $X_1$
and $X_5$.  Therefore, to satisfy the condition $X_1 \perp \chi-A-\{X_1\}|A$,
we must also include $X_5$ in $A$.  After the inclusion of $X_5$, there are
no other active paths between $X_1$ and other nodes outside of $A$ --- this
can be easily seen by trying them all.

\end{document}

