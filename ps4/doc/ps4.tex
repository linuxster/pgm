\documentclass[11pt]{article}
\usepackage{fullpage}
\usepackage{fancyhdr}

\usepackage{amsmath}
\usepackage{amssymb}
\usepackage{url}

\usepackage{listings}
\lstset{language=Python,
        basicstyle=\footnotesize\ttfamily,
        showspaces=false,
        showstringspaces=false,
        tabsize=2,
        breaklines=false,
        breakatwhitespace=true,
        identifierstyle=\ttfamily,
        keywordstyle=\bfseries,
        commentstyle=\it,
        stringstyle=\it,
    }

\usepackage[pdftex]{graphicx}

% header
\fancyhead{}
\fancyfoot{}
\fancyfoot[C]{\thepage}
\fancyhead[R]{Daniel Foreman-Mackey}
\fancyhead[L]{Probabilistic Graphical Models --- Problem Set 4}
\pagestyle{fancy}
\setlength{\headsep}{25pt}

% shortcuts
\newcommand{\Eq}[1]{Equation (\ref{eq:#1})}
\newcommand{\eq}[1]{Equation (\ref{eq:#1})}
\newcommand{\eqlabel}[1]{\label{eq:#1}}
\newcommand{\Fig}[1]{Figure \ref{fig:#1}}
\newcommand{\fig}[1]{Figure \ref{fig:#1}}
\newcommand{\figlabel}[1]{\label{fig:#1}}

% commands
\newcommand{\pr}[1]{\ensuremath{p(#1)}}
\newcommand{\bvec}[1]{\ensuremath{\boldsymbol{#1}}}
\newcommand{\dd}{\ensuremath{\, \mathrm{d}}}

\newcommand{\code}[1]{{\sffamily #1}}

\begin{document}

My code is implemented in \code{Python} and it depends on the
\code{Numpy}\footnote{\url{http://numpy.scipy.org}} package for vector
math support. The code has been tested using \code{Python v2.7} with
\code{Numpy v1.6.1} on \code{Mac OS X 10.6} but it should work with somewhat
earlier versions of \code{Python} and \code{Numpy} on any system.

All the code is implemented in \code{mplp.py} and you can run it to generate
the output files by running
\begin{lstlisting}
    python mplp.py
\end{lstlisting}
at the command line.

\section{Problem 1}

For the kitchen image, my code converged after 61 iterations and the list of
objects detected is:
\code{cupboard}, \code{dishwasher}, \code{floor}, \code{refrigerator},
\code{stove} and \code{wall}.

For the office scene, it converged after 58 iterations and detected
\code{floor} and \code{wall}.

I know that there is something wrong with my implementation because I never
reach an integrality gap of zero.

\newpage
\section{Problem 2}

For ``2dri'', I also never reached an integrality gap of zero but the best
solution that I reached (after 103 iterations), was:

\begin{lstlisting}
  1  0  0  0  0  0  0  0  0  0  0  1  2  1  0  2  0  0  2  6  0  0  0  2  0
 10  0  0  0  0  0  0  4  0  0  0  0  0  5  1  2  0  0  0  0  2  0  0  0  0
  2  0  0  2  0  0  0  0  2  3  0  1  1  0  0  1  0  4  0  0  1  0  0  0  0
  0  4  0  7  0  0  2  0  0  0  0  4  0  3 31 15  0  5  0  0  0  0  0  3  5
  0  0  4  0  1  0  0  0  0  2  0  0  0  0  2  0  0  0  2  0  0  0  0  0  1
  0  0  1  1  3  0  2  0  0  4  3  0  0  0  3  1  0  1  0  0  4  4  0  0  0
  0  0  0  2  1  0  0  0  2  0  0  0  1  3  1 10  0  0  0  2  0  0 12  1  0
  0  0  0  0  0  0  1  1  2  0  0  3  0 13  3  0  4  2  0  0  0  0  0  0  0
  0  1  0  0  0  3  0  2  0  7  0  0  0  0  0  0  2  0  0  0  0  2  0  0  0
  1  0  0  1  0  0  0  1  0  2  0  1  6  1  0  0  0 32  0  0  3  1  0  3  1
  0  0  5  0  0  0  0  0  0  0  3  0  0  0  0  0  0  0  0  1  0
\end{lstlisting}

For ``1exm'', I found the optimum after 109 iterations with an integrality
gap of 42. The best assignment was:

\begin{lstlisting}
  0  1  0  0  0  0  8  0  1  1  0  0  0  6  0  0  0  1  0  3  0 15  5  0  1
  2  0  0  0  4  1  0  1  0  0 11  3  0  0  0  0  2  3  3  0  0  1  0  3 12
  0  0  0  3  0  0  3  0  0  0  0  0  3  0  3  0  5  1  2  2  0  1 12  4  1
  0  2  0  1  0  0  0  0  0  4  0  0  0  0  7  0  1  0  0  0  3  7  0  0  0
  1  0  0  0  2  0  0  0  0  0  4  0  1  1  0  3  1  0  0  2  0 30 11  0  0
  0  0  0  0  0  0  1  2  4  0  0  3  0  0  0  0  1  1  1  1  0  0  0  0  0
  0  0  4  0  3  0  0  0  1  0  0  0  0  0  0  2  0  0  1  5  0  0  0  1  3
  0  1  0  4  0  2  0  1  0  0  2  1  0  0  0  7  4  6  0  0  3  1  2  3  0
  0  0  0  0  0  0  2  0  0  6  1  0 30  3  0  0  1  0  0  0  1  0  0 16  0
  2  0  1  0  0  0 12  0  2  1  0  4  0 35  0  8  3  0  0  0  6  0  0  0  1
  0  6  0  0  0  0  0  0  0  5  2  0  0  0  0  1  0  0  0 13  5  1  0  0  2
  0  2  0  0  0  0  0  1  0  0  0  0  0  2  0  0  2  4  0  1  2  0 34  0  1
  0  2  0  0  0  0  1  3  0  0  0  1  0  0  2  0  0  0  3  0  0  0 12  0  1
  0  0  5  2  0  0  1  0  1  0  8  4  0  2  0  0  0  0  0  0  0  0  0  0  0
  0  2  0  0  0  0  0  0  1  0  0  0  0  2  0  0  1  0  0  0  0  0  0  0  0
  0  0  0  0  3  0  0  3  0  0  0 27  7  0  0 20  2  0  0  0  0  1  0  0  0
  0  0  0
\end{lstlisting}

\end{document}

